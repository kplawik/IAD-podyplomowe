\documentclass[12pt]{article}
\usepackage[T1]{polski}
\usepackage{graphicx}
\usepackage{amsfonts}

\setlength{\textheight}{21cm}
\newcommand{\BibTeX}{{\sc Bib}\TeX}

\title{{\bf Sprawozdanie}\linebreak
Inteligentna Analiza Danych}
\author{Konrad Pławik}
\date{data oddania zadania}

\begin{document}
\clearpage\maketitle
\thispagestyle{empty}
\newpage
\setcounter{page}{1}
\section{Cel zadania}

Celem zadania było przetestowanie możliwości rozpoznawania znaków graficznych przez sieć neuronową 
typu MADALINE. W tym celu skorzystaliśmy z sieci napisanej na zajęciach z tą różnicą że predefiniowane znaki zastąpiliśmy możliwościa odczytu ich z plików PNG. Do wygenerowania plików ze znakami skorzystaliśmy ze skryptu napisanego w języku Python i bibliotek PIL oraz OpenCV.\\

\section{Wstęp teoretyczny}

Zbudowana sieć wykorzystywała model neuronu, wzorca, zestawu wzorców oraz model sieci. Model sieci miał zaimplementowane metody treningu oraz testowania. Dodatkowo także utworzenia dodatkowych funkcji i metod pomocniczych, jak np. metody do wyświetlania zawartość zestawów wzorców oraz funkcje do działań na wektorach (iloczyn wektowowy, normalizacja, wyznaczanie długości).\\

\noindent
W celu prostej automatyzacji stworzony został również skrypt tworzący obrazki z zadanymi literami wybranej czcionki o zadanym rozmiarze i stopniu zaszumienia. W głównym katalogu znajdują się skrypty powłoki bash (create\_trainset.sh oraz create\_testsets.sh) które uruchamiają w/w skrypt i generują zestawy danych: treningowy oraz testowe dla każdego z siedmiu przeprowadzonych eksperymentów. Dla pełniejszej automatyzacji stworzono jeszcze skrypt powłoki uruchamiający testy dla zbiorów testowych (run\_tests.sh).\\

\section{Eksperymenty i wyniki}

W celu przetestowania stworzonego rozwiązania przeprowadzono siedem eksperymentów dla różnych wariantów zaszumienia, różnych czcionek oraz zestawów kilku znaków. Przebieg i wyniki eksperymentów opisano w kolejnych podrozdziałach.

\subsection{Test poprawności działania}
Nim przystąpiłem do wykonywania eksperymentów dokonałem sprawdzenia poprawności działania sieci. W celu tym uruchomiłem skrypt podając dane treningowe również jako dane testowe. Zgodnie z przewidywaniami litery zostały rozpoznane poprawnie ze 100\% pewnością:

\begin{verbatim}
Network test
letter A, noise level 0% -> letter A, noise level 0%, confidence: 1.000000
letter B, noise level 0% -> letter B, noise level 0%, confidence: 1.000000
letter C, noise level 0% -> letter C, noise level 0%, confidence: 1.000000
\end{verbatim}

\subsection{Eksperyment nr 1}

Eksperyment nr 1 potraktowałem jako rozgrzewkę. Utworzyłem trzy znaki o różnym stopniu zaszumienia oraz dodałem dwa inne znaki - D oraz X.

\subsubsection{Założenia}
Zakładam że litery zostaną rozpoznane poprawnie ale poziom pewności będzie proporcjonalny do stopnia zaszumienia. Spodziewam się również że litera D zostanie rozpoznana jako B ze względu na podobieństwo krztałtu. Litera X rozstanie rozpozana jako któraś z testowych ale zapewne z niskim stopniem pewności.

\subsubsection{Przebieg}
Po wygenerowaniu zadanych danych testowych i uruchomieniu testów otrzynaliśmy następujący listing:
\begin{verbatim}
Eksperyment 1
letter A, noise level 15% -> letter A, noise level 0%, confidence: 0.916784
letter B, noise level 25% -> letter B, noise level 0%, confidence: 0.851735
letter C, noise level 50% -> letter C, noise level 0%, confidence: 0.695136
letter D, noise level 0%  -> letter B, noise level 0%, confidence: 0.981603
letter X, noise level 0%  -> letter A, noise level 0%, confidence: 0.969461
\end{verbatim}

\subsubsection{Rezultat}
Zgodnie z przewidywaniami llitery A,B i C roztały rozpoznane poprawnie a wraz ze wzrostem stopnia zaszumienia stopień pewności malał. Litera D została rozpoznana jako B w dużym stopniem pewności (zapewne ze względu na podobieństwo wyglądu) a litera X jako A (zapewne ze względu na długie skośne krawędzie) ale stopień pewności był nieznacznie niższy niż dla litery D.\\

\subsection{Eksperyment nr 2}
Zgodnie z treścią zadanie ekperyment wykorzystał zaszumienie 10\%. Literę D roztawiłem ale X usunąłem ze względu na niski stopień pewności.

\subsubsection{Założenia}
Spodziewam się poprawnego rozpoznania liter A, B i C i stopnia pewności powyżej 90\%.\\

\subsubsection{Przebieg}
Po wygenerowaniu zadanych danych testowych i uruchomieniu testów otrzynaliśmy następujący listing:
\begin{verbatim}
Eksperyment 2
letter A, noise level 10% -> letter A, noise level 0%, confidence: 0.947077
letter B, noise level 10% -> letter B, noise level 0%, confidence: 0.945833
letter C, noise level 10% -> letter C, noise level 0%, confidence: 0.947273
letter D, noise level 10% -> letter B, noise level 0%, confidence: 0.930001
\end{verbatim}

\subsubsection{Rezultat}
Zgodnie z przewidywaniami llitery A,B i C roztały rozpoznane poprawnie a stopień pewności oscylował w granicach 94\%. Litera D tak jak ostatnio - stopień pewności niższy od B. Analiza wzrokowa obrazów testowych wykazała lekkie zaszumienie a litery czytelne bez majmniejszego prablemu.\\

\subsection{Eksperyment nr 3}
Zgodnie z treścią zadanie ekperyment wykorzystał zaszumienie 30\%.
\subsubsection{Założenia}
Litery A, B i C zostaną rozpoznane poprawnie z dokładnością ok. 75\%. Litera D minimalnie niżej.
\subsubsection{Przebieg}
Po wygenerowaniu zadanych danych testowych i uruchomieniu testów otrzynaliśmy następujący listing:
\begin{verbatim}
Eksperyment 3
letter A, noise level 30% -> letter A, noise level 0%, confidence: 0.825020
letter B, noise level 30% -> letter B, noise level 0%, confidence: 0.820656
letter C, noise level 30% -> letter C, noise level 0%, confidence: 0.825726
letter D, noise level 30% -> letter C, noise level 0%, confidence: 0.810182
\end{verbatim}

\subsubsection{Rezultat}
Litery A, B i C rozpoznane poprawnie z dokładnością nieco wyższą niż się spodziewałem (ta oscylowała w ok. 82\%). Litera D rozpoznana jako C z dokładnością 81\% czyli pozostawienie jej w kolejnych iteracja jest już tylko ciekawostką. Wygenerowane pliki graficzne czytelne dla ludzkiego oka choć szum wyraźnie je utrudniał.\\

\subsection{Eksperyment nr 4}
Ekperyment wykorzystał zaszumienie 50\%.
\subsubsection{Założenia}
Zakładam poziom rozpoznania w granicach 50\%.
\subsubsection{Przebieg}
Po wygenerowaniu zadanych danych testowych i uruchomieniu testów otrzynaliśmy następujący listing:
\begin{verbatim}
Eksperyment 4
letter A, noise level 50% -> letter A, noise level 0%, confidence: 0.693884
letter B, noise level 50% -> letter A, noise level 0%, confidence: 0.691056
letter C, noise level 50% -> letter C, noise level 0%, confidence: 0.695136
letter D, noise level 50% -> letter C, noise level 0%, confidence: 0.691716
\end{verbatim}
\subsubsection{Rezultat}
Podany poziom wynosił 69\%. z tym że litera B nie została rozpoznana poprawnie (jako A). Litera D jako C (czyli jak w poprzednim przypadku). Wygenerowane pliki graficzne praktycznie nieczytelne dla ludzkiego oka.

\subsection{Eksperyment nr 5}
Ekperyment wykorzystał zaszumienie 90\%.
\subsubsection{Założenia}
Spodziewam się niepoprawności z rozpoznaniu i niskich wskazań pewności (pewnie ok. 50\%).
\subsubsection{Przebieg}
Po wygenerowaniu zadanych danych testowych i uruchomieniu testów otrzynaliśmy następujący listing:
\begin{verbatim}
Eksperyment 5
letter A, noise level 90% -> letter C, noise level 0%, confidence: 0.345270
letter B, noise level 90% -> letter A, noise level 0%, confidence: 0.335060
letter C, noise level 90% -> letter A, noise level 0%, confidence: 0.340770
letter D, noise level 90% -> letter A, noise level 0%, confidence: 0.353479
\end{verbatim}
\subsubsection{Rezultat}
Żadna z liter nie rozpozanana poprawnie. Podane współczynniki pewności ok 35\%. Naaoczne oględziny wygenerowanych plików pokazują że są już one prawie negatywami liter treningowych.

\subsection{Eksperyment nr 6}
Eksperyment dodatkowy sprawdzający jak sieć poradzi sobie z zestawami liter - jak np. Ai albo Bo.

\subsubsection{Założenia}
Ze względu brak zmienionej centryzacji spodziewam się dosyć dużej rozpoznawalności bo dodatkowe litery będą umieszczone obok zadanych (tożsamym z treningowymi).

\subsubsection{Przebieg}
Po wygenerowaniu zadanych danych testowych i uruchomieniu testów otrzynaliśmy następujący listing:
\begin{verbatim}
Eksperyment 6
letter Ai, noise level 0% -> letter A, noise level 0%, confidence: 0.989587
letter Bo, noise level 0% -> letter B, noise level 0%, confidence: 0.982996
letter Cy, noise level 0% -> letter C, noise level 0%, confidence: 0.983019
letter DT, noise level 0% -> letter B, noise level 0%, confidence: 0.961837
\end{verbatim}
\subsubsection{Rezultat}
Zgodnie z założeniami - dodatkowe litery stały obok wzorcowych dzięki czemu sieć prawidłowo rozpoznała litery dominujące (A, B, C) a dodatkowe (i, o, y, T) stanowiły jedynie "szum" dla stopnia pewności, ktory wynosił 96-98\%.\\

\subsection{Eksperyment nr 7}
Tym razem dane testowe wygenerowane zostały z użyniem czcionki consolas.ttf (poprzednie times.ttf).
\subsubsection{Założenia}
Spodziewam się błedów i dużego spadku pewności rozpozania.
\subsubsection{Przebieg}
Po wygenerowaniu zadanych danych testowych i uruchomieniu testów otrzynaliśmy następujący listing:
\begin{verbatim}
Eksperyment 7
letter A, noise level 0% -> letter A, noise level 0%, confidence: 0.967252
letter B, noise level 0% -> letter C, noise level 0%, confidence: 0.960595
letter C, noise level 0% -> letter C, noise level 0%, confidence: 0.968028
\end{verbatim}
\subsubsection{Rezultat}
Dwie litery rozpoznane poprawnie - A i C. B błednie rozpoznana jako C z zastanawiająco wysokim stopniem dokładności - 96\%.
\section{Wnioski}

Wnioski z przeprowadzonych eksperymentów dowodzą, że...


\section{Załączniki*}

Opcjonalnie, w zależności od zadania,
np. fragment kodu źródłowego.

\renewcommand\refname{Bibliografia}
\bibliographystyle{plain}
\bibliography{bibliografia}

\end{document}
