\documentclass[12pt]{article}
\usepackage[T1]{polski}
\usepackage{graphicx}
\usepackage{amsfonts}

\setlength{\textheight}{21cm}
\newcommand{\BibTeX}{{\sc Bib}\TeX}

\title{{\bf Sprawozdanie}\linebreak
Inteligentna Analiza Danych}
\author{Imię i Nazwisko Autora}
\date{data oddania zadania}

\begin{document}
\clearpage\maketitle
\thispagestyle{empty}
\newpage
\setcounter{page}{1}
\section{Cel zadania}

Opis celu zadania (proszę nie przepisywać treści instrukcji!).\\
Sprawozdanie należy wykonać na podstawie\\ 
szablonu \LaTeX-owego \texttt{sprawozdanie-wzor.tex}.

\section{Wstęp teoretyczny}

Krótki opis wykorzystywanych metod~\cite{dowolna_etykieta_artykulu}. Proszę nie umieszczać ogólnie znanych z literatury
wzorów oraz definicji. Należy podać jaka metoda została zastosowana, dlaczego oraz podać wykorzystaną literaturę (korzystając z odwołań do pozycji bibliografii~\cite{dowolna_etykieta_ksiazki}).\\
Przygotowując bibliografię należy korzystać z podanego\\ 
szablonu \BibTeX-owego \texttt{bibliografia-wzor.bib}.

\section{Eksperymenty i wyniki}

Opis wykonywanych eksperymentów. Wymagane jest ilustrowanie przeprowadzanych doświadczeń wykresami oraz tabelami.

%%%%%%%%%%%%%%%%%%%%%%%%%%%%%%%%%%%%%%%%%%%%%%%%%%%%%%%%%%%%%%%%%%%%%%%%%%%%%%%%%%%%%%%%%%%%%%%%%%%%%%%%%%%%%%%%%
% PODROZDZIAŁ PT. EKSPERYMENT NR 1 
%%%%%%%%%%%%%%%%%%%%%%%%%%%%%%%%%%%%%%%%%%%%%%%%%%%%%%%%%%%%%%%%%%%%%%%%%%%%%%%%%%%%%%%%%%%%%%%%%%%%%%%%%%%%%%%%%

\subsection{Eksperyment nr 1}

Eksperyment nr 1...\\
Identycznościowa funkcja aktywacji ma postać:
\begin{equation}
 \forall\: s \in \mathbb{R}\:\:\:\: f(s) = s
 \label{równanie funkcji identycznościowej}
\end{equation}
Jak widać z definicji (\ref{równanie funkcji identycznościowej}) funkcja ta...

\subsubsection{Założenia}

\subsubsection{Przebieg}
\newpage

\subsubsection{Rezultat}

Rezultaty badań eksperymentalnych przedstawione są w Tab. \ref{wyniki eksperymentu pierwszego}.
\begin{table}[h!]
 \caption{Rezultaty eksperymentu nr 1}
 \centering
 \vspace{0.2cm}
 \begin{tabular}{c c c c}
  \hline\hline\\[-0.4cm]
  \textbf{Przypadek} & \textbf{Metoda 1} & \textbf{Metoda 2} & \textbf{Metoda 3}\\[0.1cm]
  \hline
  \textbf{1} & 50 & 837 & 970  \\
  \textbf{2} & 47 & 877 & 230  \\
  \textbf{3} & 31 &  25 & 415  \\
  \textbf{4} & 35 & 144 & 2356 \\
  \textbf{5} & 45 & 300 & 556  \\ [0.1cm]
  \hline
 \end{tabular}
 \label{wyniki eksperymentu pierwszego}
\end{table}

\noindent Jak widać w Tab. \ref{wyniki eksperymentu pierwszego}...\newline
Graficzna interpretacja wyników z Tab. \ref{wyniki eksperymentu pierwszego} 
przedstawiona jest na wykresie Rys. \ref{rysunek do eksperymentu pierwszego} gdzie można zauważyć, że...
\begin{figure}[h!]
 \centering
 \includegraphics[width=9.3cm]{wykres.pdf}
 \vspace{-0.3cm}
 \caption{Wykres dla wyników eksperymentu pierwszego}
 \label{rysunek do eksperymentu pierwszego}
\end{figure}

\noindent Jak widać z wykresu Rys. \ref{rysunek do eksperymentu pierwszego}...\newline

%%%%%%%%%%%%%%%%%%%%%%%%%%%%%%%%%%%%%%%%%%%%%%%%%%%%%%%%%%%%%%%%%%%%%%%%%%%%%%%%%%%%%%%%%%%%%%%%%%%%%%%%%%%%%%%%%
% PODROZDZIAŁ PT. EKSPERYMENT NR 2 
%%%%%%%%%%%%%%%%%%%%%%%%%%%%%%%%%%%%%%%%%%%%%%%%%%%%%%%%%%%%%%%%%%%%%%%%%%%%%%%%%%%%%%%%%%%%%%%%%%%%%%%%%%%%%%%%%

\newpage
\subsection{Eksperyment nr 2}

Eksperyment nr 2 polegał na...\\
Sigmoidalna funkcja aktywacji ma postać:
\begin{equation}
 \forall\: s \in \mathbb{R}\:\:\:\:\:\: f(s) = \frac{1}{\:\:\:1 + e^{-\beta \cdot s}\:}\:,
 \:\:\:\:\textnormal{gdzie}\:\:\beta \in \mathbb{R}_{+}
 \label{równanie funkcji sigmoidalnej}
\end{equation}
Jak widać z równania definicyjnego (\ref{równanie funkcji sigmoidalnej}) 
funkcja\footnote{ang. \textit{sigmoidal function} lub \textit{unipolar function}}
ta ma wykres przedstawiony 
na rysunku Rys. \ref{funkcja sigmoidalna}, gdzie paramater $\beta$ ...
\begin{figure}[h!]
 \centering
 \includegraphics[width=7.3cm]{funkcja.png}
 \vspace{-0.1cm}
 \caption{Wykres funkcji sigmoidalnej}
 \label{funkcja sigmoidalna}
\end{figure}

\subsubsection{Założenia}

\subsubsection{Przebieg}

\subsubsection{Rezultat}

Rezultaty badań eksperymentalnych przedstawione są w Tab. \ref{wyniki eksperymentu drugiego}.
\vspace{-0.5cm}
\begin{table}[h!]
 \caption{Rezultaty eksperymentu nr 2}
 \centering
 \vspace{0.2cm}
 \begin{tabular}{c c c }
  \hline\hline\\[-0.4cm]
  \textbf{Przypadek} & \textbf{Metoda 1} & \textbf{Metoda 2}\\[0.1cm]
  \hline
  \textbf{1} & 50 & 837 \\
  \textbf{2} & 47 & 877 \\
  \textbf{3} & 45 & 300 \\ [0.1cm]
  \hline
 \end{tabular}
 \label{wyniki eksperymentu drugiego}
\end{table}

\noindent Jak widać w Tab. \ref{wyniki eksperymentu drugiego}...\newline
Wyniki w Tab. \ref{wyniki eksperymentu drugiego} świadczą o tym, że...

%%%%%%%%%%%%%%%%%%%%%%%%%%%%%%%%%%%%%%%%%%%%%%%%%%%%%%%%%%%%%%%%%%%%%%%%%%%%%%%%%%%%%%%%%%%%%%%%%%%%%%%%%%%%%%%%%
% PODROZDZIAŁ PT. EKSPERYMENT NR N 
%%%%%%%%%%%%%%%%%%%%%%%%%%%%%%%%%%%%%%%%%%%%%%%%%%%%%%%%%%%%%%%%%%%%%%%%%%%%%%%%%%%%%%%%%%%%%%%%%%%%%%%%%%%%%%%%%

\subsection{Eksperyment nr n}
Eksperyment nr n zakładał, iż...\\
Dla dowolnej liczby $N \in \mathbb{N}$ funkcję 
$F_{N}:\mathbb{C}^{\:N}\!\rightarrow\mathbb{C}^{\:N}$
zdefiniowaną w następujący sposób:
\vspace{-0.4cm}
\begin{equation}
 \forall\:\mathbf{x} \in \mathbb{C}^{\:N}\:\:\:\:\:
 \forall\:k \in \{\:0,\dots,N - 1\:\!\}\:\:\:\:\:
 F_{N}\:\!(\:\mathbf{x}\:)_{k}\:\stackrel{\Delta}{=}\:
 \frac{1}{\sqrt{N}}\:
 \sum_{n = 0}^{N - 1}\:x_{n}\:\cdot\:
 e^{-j 2 \pi n k / N}
 \label{równanie dyskretnej transformaty Fouriera}
\end{equation}
nazywamy $N$ -- punktowym prostym jednowymiarowym dyskretnym przekształceniem Fouriera.
Na Rys. \ref{FFT} przedstawiono szybki algorytm obliczania dyskretnego 
przekształcenia Fouriera\footnote{ang. \textit{Fast Fourier Transform}}.
\begin{figure}[h!]
 \centering
 \includegraphics[width=0.8\linewidth]{transformata.pdf}
 \vspace{-0.1cm}
 \caption{Szybkie przekształcenie Fouriera}
 \label{FFT}
\end{figure}

\subsubsection{Założenia}

\subsubsection{Przebieg}

\subsubsection{Rezultat}
\newpage

\section{Wnioski}

Wnioski z przeprowadzonych eksperymentów dowodzą, że...


%%%%%%%%%%%%%%%%%%%%%%%%%%%%%%%%%%%%%%%%%%%%%%%%%%%%%%%%%%%%%%%%%%%%%%%%%%%%%%%%%%%%%%%%%%%%%%%%%%%%%%%%%%%%%%%%%
% PODROZDZIAŁ PT. ZALACZNIKI
%%%%%%%%%%%%%%%%%%%%%%%%%%%%%%%%%%%%%%%%%%%%%%%%%%%%%%%%%%%%%%%%%%%%%%%%%%%%%%%%%%%%%%%%%%%%%%%%%%%%%%%%%%%%%%%%%

\section{Załączniki*}

Opcjonalnie, w zależności od zadania,
np. fragment kodu źródłowego.


%%%%%%%%%%%%%%%%%%%%%%%%%%%%%%%%%%%%%%%%%%%%%%%%%%%%%%%%%%%%%%%%%%%%%%%%%%%%%%%%%%%%%%%%%%%%%%%%%%%%%%%%%%%%%%%%%
% BIBLIOGRAFIA
%%%%%%%%%%%%%%%%%%%%%%%%%%%%%%%%%%%%%%%%%%%%%%%%%%%%%%%%%%%%%%%%%%%%%%%%%%%%%%%%%%%%%%%%%%%%%%%%%%%%%%%%%%%%%%%%%

\renewcommand\refname{Bibliografia}
\bibliographystyle{plain}
\bibliography{bibliografia}

\end{document}
